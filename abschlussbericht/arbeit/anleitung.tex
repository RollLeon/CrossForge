% vim: set ts=4 sw=4 smartindent expandtab textwidth=100:

\documentclass[a4paper, 11pt]{article}

\usepackage{fontspec}                                   % eigene (TTF) Schriftarten
\usepackage{polyglossia}
    \setdefaultlanguage[babelshorthands=true]{german}   % Silbentrennungsmuster
\usepackage{microtype}                                  % Textsatzverbesserung
\usepackage[german]{selnolig}                           % falsche Ligaturen trennen
\usepackage{typearea}                                   % Seitenschnitt 

\usepackage[backend=biber,
            maxbibnames=99,
            style=alphabetic,
            citestyle=alphabetic]
           {biblatex}                                   % für das Literaturverzeichnis
\usepackage{booktabs}                                   % tolle Linien in Tabellen
\usepackage[font=small,labelfont=bf]{caption}           % Bild-/Tabellenunterschriften leicht
                                                        %   anpassen
\usepackage{csquotes}                                   % landesspezifische Anführungszeichen
\usepackage{icomma}                                     % Komma als Dezimaltrenner richtig behandeln
\usepackage{ltablex}                                    % lange variabel breite Tabellen

\addbibresource{literatur.bib}                          % Bibliographiedatenbank laden

\parindent 2em                                          % Absatzerstzeileneinzüge
                                                        %   (1em = Breite des M, 1ex = Höhe des x)
\parskip 0.5em                                          % Absatzabstäende

\title{Anleitung für Wissenschaftliche Arbeiten}
\author{Alexander Adam}
\date{\today}

\begin{document}

\maketitle

\begin{abstract}
Dieses Dokument beschreibt einige der Grundzüge wissenschaftlichen Arbeitens. Es gibt einen Überblick über die Struktur und die Vorgehensweisen bei der Erstellung von unter anderem Praktikums-, Seminar, Bachelor- und Masterarbeiten.
\end{abstract}

\tableofcontents

%---------------------------------------------------------------------------------------------------
\section{Einleitung}

Früher oder später muss jeder Student einmal eine schriftliche Ausarbeitung erledigen, spätestens die Bachelorarbeit. An der Universität sollten solche Schriftstücke eine spezielle Form einhalten, die sich im wissenschaftlichen Umfeld bewährt und durchgesetzt hat.

Im Folgenden sollen verschiedene Aspekte wissenschaftlicher Arbeiten beleuchtet werden. Zusätzlich gibt es eine kleine Einführung in das, im Paket enthaltene, \LaTeX-Template.

%---------------------------------------------------------------------------------------------------
\section{Äußere Form}

Die äußere Form einer wissenschaftlichen Arbeit hängt stark vom Empfänger ab. Einige Formanforderungen orientieren sich eher an einer guten Korrigierbarkeit (z.\,B. durch exorbitante Seitenränder und Zeilenabstände) als guter Lesbarkeit und Form. Der Autor dieser Anleitung will sich aber auch nicht anmaßen, der Weisheit letzten Schluss gefunden zu haben, zumindest aber \LaTeX.

\LaTeX{} ist im universitären Umfeld weit verbreitet und eignet sich für alle Arten von Dokumenten, auch Folien. Die guten Voreinstellungen ermöglichen es auch einem Laien ein zumindest optisch ansprechendes Dokument zu erstellen. Der Autor ist überzeugt, dass es sich lohnt sich mit \LaTeX{} auseinanderzusetzen, wird es aber nicht verpflichtend vorgeben.

Die folgenden Ausführungen beschreiben meist Selbstverständlichkeiten, die Erfahrung zeigt jedoch, dass auch diese besser nochmals explizit ausgeführt werden sollten.

% -- -- -- -- -- -- -- -- -- -- -- -- -- -- -- -- -- -- -- -- -- -- -- -- -- -- -- -- -- -- -- -- --
\subsection{Textgestaltung}

\subsubsection{Textbild}

Es geht hier um das, was sichtbar wäre, würde aller Text durch schwarze Balken ersetzt. Es gilt also den Seitenschnitt (vereinfacht Seitenränder), Absatzabstand, Absatzeinzüge und Zeilenabstände zu bestimmen. \LaTeX{} hat hier schon sehr gute Vorgaben für englische Text (ja, da gelten andere typographische Regeln), für deutsche Texte gibt es die folgenden Anpassungen (die auch in englischen Texten gut aussehen).

Absätze sollten im Blocksatz formatiert sein. Damit der gut funktioniert, ist die automatische Silbentrennung empfehlenswert (die ist in \LaTeX{} standardmäßig aktiv). Der Zeilenabstand sollte einzeilig sein. Absätze werden voneinander durch einen Abstand (\texttt{\textbackslash parskip}) getrennt. An ihrem Anfang kann ein Erstzeileneinzug (\texttt{\textbackslash parindent}) sein.

Der Seitenschnitt ist eine Wissenschaft für sich. \LaTeX-Nutzer können hier auf das Paket \texttt{typearea} vertrauen, welches die Seitenränder nach bewährten Methoden einrichtet. Es ist wichtig zu wissen, dass hier auch noch Kopf- und Fußzeilen sowie Platz für Randbemerkungen berücksichtigt werden.

%- - - - - - - - - - - - - - - - - - - - - - - - - - - - - - - - - - - - - - - - - - - - - - - - - -
\subsubsection{Schrift}

Schrift besteht aus vielen Facetten. Die größte Entscheidung wird aber wohl die Schriftart betreffen. Als Faustregel kann man hier sagen, dass für alle gedruckten Dokumente, eine serifenbehaftete Schriftart (wie in diesem Dokument) zu bevorzugen ist. Auf Papier liest sich dies sehr angenehm. Für Bildschirm optimierte Dokumente sollten allerdings auf eine serifenlose Schriftart setzen (beispielsweise Arial oder Helvetica).

Für Hervorhebungen empfiehlt sich das Makro \texttt{\textbackslash emph} zu nutzen. Dieses hebt den Text abhängig von der Umgebung hervor. In kursiven Textteilen nicht-kursiv, in nicht-kursiven Textteilen kursiv.

%- - - - - - - - - - - - - - - - - - - - - - - - - - - - - - - - - - - - - - - - - - - - - - - - - -
\subsubsection{Verzierungen}

Verzierungen wie Farben oder Schmuckschriftarten sollten sehr sparsam (wenn überhaupt) eingesetzt werden. Spätestens beim kostengünstigen Schwarz-Weiß-Druck gehen Farben verloren und werden durch gerasterte Grautöne ersetzt, die häufig nicht gut lesbar sind.

% -- -- -- -- -- -- -- -- -- -- -- -- -- -- -- -- -- -- -- -- -- -- -- -- -- -- -- -- -- -- -- -- --
\subsection{Formalien}
\subsubsection{Bilder und Tabellen}

Bilder und Tabellen sind oft das Mittel der Wahl um Sachverhalte besser zu verdeutlichen. In wissenschaftlichen Arbeiten haben alle Bilder und Tabellen (mit sehr wenigen Ausnahmen) eine Nummer und tauchen im jeweiligen Abbildungs- oder Tabellenverzeichnis auf.

Für die Beschriftung gilt: Tabellen haben Tabellenüberschriften, Bilder haben Bildunterschriften. In \LaTeX{} wird dies einfach dadurch erreicht, dass einmal das \texttt{\textbackslash caption}-Makro vor bzw. nach dem Inhalt kommt.

Da in \LaTeX{} Bilder und Tabellen bevorzugt in Fließumgebungen gesetzt werden, also das Satzsystem selbst entscheidet, wo das Bild letztlich erscheint, müssen alle Bilder und Tabellen im Text verankert werden. Das heißt, dass es für jedes derartige Objekt eine Referenz im Text gibt, wie dies zum Beispiel hier mit \tablename~\ref{tab:beispieltabelle} geschieht. Wenn der Inhalt nicht die ganze Seite ausfüllt, so ist er zentriert zu setzen.

\begin{table}[tbhp]
    \caption{Beispieltabelle}
    \centering{
        \begin{tabularx}{0.5\linewidth}{lX}
        \toprule
            Nr.& Text           \\
        \midrule
            1. & Erste Zeile    \\
            2. & Zweite Zeile   \\
        \bottomrule
        \end{tabularx}
    }
    \label{tab:beispieltabelle}
\end{table}

%- - - - - - - - - - - - - - - - - - - - - - - - - - - - - - - - - - - - - - - - - - - - - - - - - -
\subsubsection{Verzeichnisse}

Inhalts-, Abbildungs-, Tabellen- und Abkürzungsverzeichnis kommen vor den eigentlichen Inhalt, Literaturverzeichnis und Index hinter den Inhalt.

%- - - - - - - - - - - - - - - - - - - - - - - - - - - - - - - - - - - - - - - - - - - - - - - - - -
\subsubsection{Nummerierung}

Alles, einschließlich der vorangestellten Verzeichnisse, bekommt eine römische Nummerierung. Im bereitgestellten Template sorgen hierfür bereits die Makros \texttt{\textbackslash frontmatter}, \texttt{\textbackslash mainmatter} und \texttt{\textbackslash backmatter}. Diese sind nur in der Dokumentklasse \texttt{book} verfügbar. Der Hauptteil und alles folgende wird normal arabisch nummeriert.

%- - - - - - - - - - - - - - - - - - - - - - - - - - - - - - - - - - - - - - - - - - - - - - - - - -
\subsubsection{Literaturverweise}

Bei Literaturverweisen, sei es um wörtliche Zitate zu kennzeichnen oder einfach nur weiterführende Informationen anzugeben, gibt es eine Vielzahl von Möglichkeiten. Wie für so vieles, gibt es in Deutschland hierfür eine Norm, die \enquote{DIN ISO 690:2013-10}. Im Prinzip ist allerdings die Form der Quellenangabe recht frei und soll letztlich nur dafür sorgen, dass
\begin{enumerate}
    \item der Leser erkennt, dass es sich nicht um die Idee des Autors handelt
    \item weiterführende Literatur direkt angegeben ist und nicht erst mühsam gesucht werden muss.
\end{enumerate}

Im Template ist ein Literaturverzeichnis vorhanden. Zum Zitieren wird das Makro \texttt{\textbackslash cite} verwendet, welches einen Schlüssel aus der Literaturdatenbankdatei als Parameter erhält.

%---------------------------------------------------------------------------------------------------
\pagebreak[4]
\section{Innere Form}

Hier soll kurz umrissen werden, aus welchen Teilen eine wissenschaftliche Arbeit besteht.

% -- -- -- -- -- -- -- -- -- -- -- -- -- -- -- -- -- -- -- -- -- -- -- -- -- -- -- -- -- -- -- -- --
\subsection{Einleitung}
% vim: set ts=4 sw=4 smartindent expandtab textwidth=100:

Die Einleitung dient der Vorstellung des Themas. Hier sollte für \emph{jeden} verständlich dargestellt werden, um was es geht. Es geht darum, eine Vorstellung davon zu vermitteln, was der Autor in dieser Arbeit geleistet hat.

Am Ende der Einleitung sollte ein grober Überblick über die Struktur der Arbeit gegeben werden.


% -- -- -- -- -- -- -- -- -- -- -- -- -- -- -- -- -- -- -- -- -- -- -- -- -- -- -- -- -- -- -- -- --
\subsection{Stand der Technik}
\input{tex/stand_der_technik.tex}

% -- -- -- -- -- -- -- -- -- -- -- -- -- -- -- -- -- -- -- -- -- -- -- -- -- -- -- -- -- -- -- -- --
\subsection{Konzept}
% vim: set ts=4 sw=4 smartindent expandtab textwidth=100:

\section{Vorbedingungen}

Gemeinsam mit unserem Betreuer haben wir uns darauf geeinigt CrossForge zu verwenden. Die Engine unterstützt Windows, Linux und WebAssembly und benutzt nur OpenGL als Grafik API. Sie unterstützt physikalisch basiertes Rendering und Materialien, Deferred und Forward Rendering Pipelines und ein Shadersystem. Aber besonders wichtig für uns sind der Szenengraph, die Skelettanimationen und Text Rendering.

\section{Architektur und benötigte Systeme}

Aus dem MVP-Szenario und der Engine Wahl ergeben sich mehrere Anforderungen, die über verschiedene Systeme gelöst werden können. Die Roboter sind selbständige Agenten, die Entscheidungen treffen und diese auch ausführen sollen. Dieses Problem löst das Multiagentensystem. Sie müssen ihren Weg zur nächsten durstigen Pflanze finden, dafür ist das Navigationssystem zuständig. Die Roboter müssen die Pflanze gießen, was durch ein Partikelsystem visualisiert werden soll. Der Spieler und die Roboter sollen sich auf verschieden Plattformen bewegen und von einer zur nächsten laufen können. Zusätzlich sollen die Pflanzen verschiebbar sein. Um diese beiden Sachen zu realisieren, wird ein Physiksystem benötigt.
Der Spieler soll sich mit den Robotern unterhalten können, was durch das Dialogsystem abgedeckt wird.
Da CrossForge keinen eigenen Szenen Editor hat und die MVP Szene schon zu komplex ist, um diese mit Programmcode zu beschreiben, ist ein Szenen Editor nötig.

Um alle Systeme möglichst flexibel und unabhängig voneinander zu gestalten, haben wir uns für das Entity Component Pattern entschieden.

\section{Entity Component System}

\section{Partikelsystem}


\section{Multiagentensystem}

Das Multiagentensystem ist das Herz unseres Projektes da das Verhalten von allen belebten und unbelebten Agenten von diesem System gesteuert werden sollen. Man unterscheidet zwischen Zentraler KI und Agentenbasierter KI. Bei Zentraler KI werden die Entitäten von einem externen, globalen und allwissenden System gesteuert. Individuen haben deswegen keine Kontrolle über ihre eigenen Handlungen. Zentrale KI wird sehr oft eingesetzt, weil sich damit Gruppendynamiken und taktische Vorgehensweisen einfacher implementieren lassen. Bei Agentenbasierter KI treffen die Agenten unabhängige und individuelle Entscheidungen basierend auf den Informationen, die ihnen bereit stehen. Es gibt zwar trotzdem globale Informationen, die aber nicht dafür missbraucht werden dürfen die Handlungen eines Agenten zu diktieren. Wir haben uns für Agentenbasierte KI entschieden, weil sie für unser Szenario leichter umzusetzen ist. Die Hoffnung ist, dass die Akteure dadurch natürliche Verhaltensweisen und Interaktionen zur Schau stellen.

In verschiedenen GDC \cite{YouTube_2019}\cite{YouTube_2022}\cite{YouTube_2023} Talks wurde empfohlen, ein KI-Systen in mehreren Schichten aufzubauen:

\begin{itemize}
\item Sensoren
\item Entscheidungsfindung
\item Entscheidungsausführung
\item Bewegungssteuerung
\end{itemize}

\subsection{Sensoren}

Sensoren sind ein Querschnittskonzept und tauchen deshalb in allen Ebenen auf. Sie filtern Informationen aus der Umgebung und stellen diese der Schicht bereit, in der sie eingesetzt werden. Zusätzlich können Informationen über Blackboards mit anderen Agenten geteilt werden.

\subsection{Entscheidungsfindung}

Für die Entscheidungsfindung standen die folgenden Algorithmen in der näheren Auswahl: Beliefs, Desires, Intentions (BDI), Goal Oriented Action Planning (GOAP) und Finite State Machines (FSM). Am Ende haben wir uns für Finite State Machines entschieden, weil das das einfachste Verfahren war und vorerst für die Roboter ausreicht.

Eine State Machine ist ein gerichteter Graph mit limitierter Anzahl an Stati und Aktionen. Der Agent wechselt von einem Status in den nächsten, falls eine Bedingung erfüllt ist. In unserer Simulation sollen FSMs für die Roboter eingesetzt werden, um die Stati \textit{Pflanzen gießen}, \textit{Dialog mit Spieler}, \textit{Spieler folgen}, etc abdecken zu können. 

\subsection{Entscheidungsausführung}

Wenn der Agent eine Entscheidung getroffen hat, dann lässt sich diese Entscheidung meistens in weitere Teilprozesse zerlegen. Wenn ein Roboter zum Beispiel entschieden hat, dass er jetzt Pflanzen gießt, dann muss er:

\begin{itemize}
\item eine durstige Pflanze finden
\item zur Pflanze hin fahren
\item die Gießkanne zur Pflanze ausrichten
\item und schließlich die Pflanze gießen
\end{itemize}

Diese Sequenz von Handlungen lässt sich nur sehr schlecht mit FSMs darstellen, weswegen dieser Nachteil durch Behaviour Trees ausgeglichen werden soll. Behaviour Trees sind sehr gut darin solche Sequenzen darzustellen oder sogar nebenläufige Handlungen zu beschreiben. Ihr Nachteil ist jedoch, dass sie nur sehr schwierig auf Übergänge von einen Status in den nächsten reagieren können. Ein Beispiel hierfür ist, dass der Spieler den Roboter anspricht und ihn über Pflanzen befragt. Um das mit Behaviour Trees festzustellen, müssen überall Monitore eingebaut werden, die dieses Ereignis erkennen und behandlen. Das macht den Baum unübersichtlich und nicht wartbar. Behaviour Trees und State Machines ergänzen sich also sehr gut.

In Behaviour Trees werden Handlungen durch Knoten beschrieben. Die Knoten können die Stati: \textit{Laufend}, \textit{Fehler} oder \textit{Abgeschlossen} haben.
Knoten können dabei über die Bearbeitung ihrer Kindknoten entscheiden. Bei einem Sequenzknoten werden die Kinder von links nach rechts abgearbeitet. Nur wenn der Knoten den Status \textit{Abgeschlossen} hat, wird der rechte Geschwisterknoten aufgerufen. Wenn der Status \textit{Laufend} ist, dann wird der Knoten solange bearbeitet, bis der Status \textit{Abgeschlossen} oder \textit{Fehler} erreicht ist. Bei dem Status \textit{Fehler} wird die Abarbeitung abgebrochen und dieser nach oben propagiert. Der Elternknoten kann dann entscheiden, wie dieser Fehlerstatus behandelt wird.
Der Fallbackknoten behandelt den Fehler zum Beispiel, indem er den ersten Kindknoten findet, der nicht fehlschlägt und somit erfolgreich ausführt. Nur wenn alle Kindknoten fehlschlagen, ist der Status des Fallbackknotens \textit{Fehler}.

\subsection{Bewegungssteuerung}

Die Ebene der Bewegungssteuerung ist dafür verantwortlich die Entscheidungen in Beschleunigung, Geschwindigkeit und Rotation umzuwandeln. Dafür haben wir uns Steering Behaviour \cite{SteeringBehaviour} näher angeschaut. Es gibt verschiedene Bewegungsmuster:

\begin{itemize}
\item Seek
\item Wander
\item Collision Avoidance
\item Queue
\item ...
\end{itemize}

Für uns ist vor allem Collision Avoidance in Verbindung mit Seek interessant.

\section{Navigationssystem}

In der Mitte der MVP-Szene befindet sich die Wendeltreppe, weswegen die Roboter nicht immer den direkten Weg zur Pflanze fahren können, weil sie dieses Hindernis beachten müssen. Das Navigationssystem übernimmt die Aufgabe einen Pfad von einem Startpunkt zum Zielpunkt zu finden. Um Zeit zu sparen und uns auf die Hauptaufgabe konzentrieren zu können haben wir uns entschieden die Bibliothek Recast und Detour einzusetzen. Recast berechnet das Navigationsmesh aus der statischen Geometrie und Detour findet zur Laufzeit einen Pfad auf diesem Navigationsmesh. Da das Navigationsmesh nicht dynamisch angepasst werden muss, reicht es aus, wenn man es einmal vor der Kompilierung berechnet, speichert und dann in CrossForge lädt. Die Bibliothek stellt hierfür ein Beispielprogramm bereit, was wir nutzen.

\section{Physiksystem}

Die Entities und der Spieler sollen mit der statischen Szenengeometrie und sich selber kollidieren. Um diese Kollisionen zu erkennen und aufzulösen, ist das Physiksystem nötig. Dafür möchten wir die Bullet Bibliothek einsetzen, da Linus schon einmal den Separating Axis Theorem Algorihmus in 2D implementiert hat und das sich als sehr Zeitaufwändig und Fehleranfällig herausgestellt hat. 


\section{Szenen Editor}

Wir benötigen einen Szenen Editor, weil schon bei einer geringen Anzahl an platzierten Entitäten der Code unübersichtlich und schwer zu warten ist. Als alternative könnten wir die Agenten und Pflanzen prozedural platzieren, was aber ein zu komplexes Gebiet wäre und damit den Rahmen sprengen würde. Selber einen Szenen Editor von Grund auf neu zu schreiben ist keine Option, weil auch das eine riesige Aufgabe ist. Deswegen haben wir uns dazu entschieden die Open Source Software Blender zu verwenden. Blender ist ein mächtiger 3D-Editor den man über Addons erweitern kann. Somit haben wir zwei Möglichkeiten, unsere Szenen zu erstellen und in CrossForge zu laden. Wir können die Szene als GLTF-Datei exportieren und dann die vorhandenen Funktionen in CrossForge erweitern um aus der GLTF-Datei die einzelnen Transformationen für die Entitäten zu extrahieren. Oder wir schreiben ein Addon, welches die Szene nach unseren Anforderungen exportiert.

\subsection{GLTF als Szenenbeschreibung}

Khronos Group veröffentlichte am 19.10.2015 das offene GLTF Format, um dreidimensionale Szenen und Modelle darzustellen. In dem Format können Szenen mit ihren Knoten, Kamerainformationen, Animationen, Texturen, Materialien und natürlich auch Modellinformationen gespeichert werden. CrossForge unterstützt über die Assimp Bibliothek verschiedene Dateiformate, unter anderem auch GLTF.

Die Idee ist, dass alle Entities einer bestimmten Namenskonvention folgen. Die Szene wird als ganzes in das GLTF-Format exportiert und mit Assimp geladen. Assimp hat einen eigenen Szenengraph, den man traversieren kann und wenn man an einem Knoten mit bestimmer Namenskonvention angelangt ist, weiß man, dass es sich um ein Entity handelt. Die Transformation des Knotens wird dann zur Transformation des Entities. Alle Kindknoten können dann zu einem Modell zusammengefasst werden und für die Geometriekomponente des Entities benutzt werden.

Dieser Ansatz hat den Vorteil, dass er Editoragnostisch ist und wir können vorhandenes Wissen über Assimp nutzen. Leider hat diese Variante aber auch große Nachteile: man muss die Änderungen ziemlich tief in CrossForge vornehmen, was sehr viele Code Änderungen nach sich zieht. Der Szenenersteller muss die Knoten im Editor korrekt benennen, weil man sie in CrossForge sonst nicht mehr erkennen kann.

\subsection{Eigenes Format zur Szenenbeschreibung}

Wir können ein Blender Plugin schreiben, was die Transformationen der einzelnen Entities in eine JSON-Datei exportiert. Das Problem ist, dass Blender erkennen muss, was statische Geometrie ist und was Entities sind. Es gibt aber eine Funktion um externe .blend Dateien in die aktuelle zu linken. Jetzt kann man die Regel aufstellen, dass alles, was gelinkt ist ein eigenes Entity ist. Die Vorteile sind, dass keine Eingriffe ins Engine-Innere nötig sind. Da man das Dateiformat selber bestimmen kann, ist die Implementierung auf CrossForge Seite auch sehr simpel. Der Nachteil ist, dass noch niemand von uns ein Blender Addon geschrieben hat und wir deswegen noch keine Erfahrung in diesem Bereich haben.

\section{Dialogsystem}

\section{Modelle und Szene}

Die Szene und die dazu gehörigen Modelle sind ein essenzieller Teil, um alle anderen Funktionen überhaupt darstellen zu können. Natürlich kann man dafür auch jedes x-beliebige Modell nutzen, wie einen einfachen Würfel. Allerdings hatten wir ein klares Ziel vor Augen, was wir ungefähr darstellen wollen. Wie schon im Abschnitt >Motivation< erklärt, besteht unser MVP aus zwei Plattformen mit jeweils einem Roboter und dazu einige Pflanzen.
Der erste und wichtigste Schritt war sich erst einmal zu überlegen, wie unsere Roboter aussehen sollen, da diese unsere ‚Hauptakteure‘ sind. Da die Roboter auch mit dem Spieler agieren, sollten diese eigentlich erst ein eher menschlicheres Design bekommen. Für den Anfang allerdings, sollte es recht einfach gehalten werden. Daher habe ich mich für ein eher ‚simples und knuffiges‘ Design entschieden und verschiedene Überlegungen skizziert. Zusätzlich habe ich mir zum Design noch einige Randnotizen zu eventuellen Funktionen gemacht. >BILD EINFÜGEN< Eine weitere Idee war es, auch um einige Zusatzprogrammierungen zu vermeiden, dass der Roboter schwebt und nicht fährt. Allerdings war dies zu zukunftsbasiert und daher erstmal eher abgelehnt. Bei der finalen Skizze habe ich dann noch einmal das Radsystem überdacht und dieses im Nachhinein noch abgeändert.  >BILD EINFÜGEN< 
Die nächsten Überlegungen gingen an die Pflanzen. Bei diesen habe ich eher etwas spontan, ohne Skizzen, gearbeitet. Es sollte sich um eine einfache Pflanze und eine eher exotische Pflanze handeln, sodass auch etwas Variation darin steckt.
Ebenso viel Aufwand im Konzept wie die Roboter, haben die Plattformen und die Szene allgemein eingenommen. Bei diesen habe ich ebenso einige Ideen skizziert und im Nachhinein abgesprochen. Anfangs habe ich direkt zukunftsbasiert gedacht und eher ausgebautere Szenen skizziert. >BILD EINFÜGEN< Speziell ans MVP basiert minimierte ich dann die Skizzen und überlegte mir eine ganz andere Form, was letztlich nach einigen Änderungen auch die finale Form wurde. >BILD EINFÜGEN<
Im späteren Verlauf forderten die Modelle natürlich auch Texturen. Zu diesem Punkt gehe ich dann im Teil >Implementierung< nochmal näher ein.


% -- -- -- -- -- -- -- -- -- -- -- -- -- -- -- -- -- -- -- -- -- -- -- -- -- -- -- -- -- -- -- -- --
\subsection{Implementierung}
% vim: set ts=4 sw=4 smartindent expandtab textwidth=100:

\section{Entity Component System}
Zu Beginn der Implementierung wurde über verschiedene Entity Component Systeme (ECS) recherchiert. Die folgenden Kriterien wurden bei der Auswahl eines passenden ECS berücksichtig. 
\begin{itemize}
	\item Aktive Entwicklung
	\item Datum des letzten Updates
	\item Unterstützte Sprachen
	\item Qualität der Dokumentation
	\item Performance
	\item Unterstützung des Open-Source-Paketmanager vcpkg
\end{itemize}
Bei der Recherche wurden, die folgenden zwei ECS, Flecs und EnTT, in Betracht gezogen. Letztendlich fiel die Wahl auf Flecs, da dieses besser dokumentiert ist und das letzte Update aktueller war. 
Im Anschluss an die Recherche wurde Flecs mit Hilfe von vcpkg eingebunden. Im Laufe des Projekts wurden die Entities Spieler, Pflanzen und Roboter angelegt und erhielten Komponenten, wie beispielsweise Fahr-, Spieler- und Pflanzenkomponente. So enthält zum Beispiel die Positionskomponente Informationen über die Translation, Rotation, Skalierung, Translationsdelta, oder Rotationsdelta. Die Pflanzenkomponente speichert, wie viel Wasser sich in der Pflanze befindet und wie viel maximal in den Topf passt. Und die Spielerkomponente hat Informationen über den Gamestate, die Kamera, Tastatur und Maus. Des Weiteren wurde die Steuerung der Kamera angepasst. Es wurde die Maus in dem Fenster gefangen, wodurch es nun möglich ist die Kamera zu bewegen, ohne vorher eine Maustaste drücken zu müssen. Außerdem wurde die Bewegungsgeschwindigkeit des Spielers in Bezug auf das Gehen und Rennen angepasst.
Zudem wurden Systeme, wie das Fahrsystem, Pflanzensystem, Gießsystem und noch viele mehr umgesetzt. Das Pflanzensystem, reduziert bei jedem Aufruf den Wasserstand aller Pflanzen, welches alle Objekte sind, die eine Pflanzenkomponente besitzen. Der Stand kann dabei jedoch nicht negativ werden. Ein weiteres Beispiel ist das „FindePflazensystem“, welches aus allen Pflanzen diejenige aussucht, welche am wenigsten Wasser hat und diese als neues Ziel festlegt. 
Da bei der Implementierung einige Komponenten und Systeme erzeugt werden, ist es wichtig sicherzustellen, dass man den Überblick über diese behält. Aus diesem Grund wurden im Laufe des Projekts Aufräumarbeiten durchgeführt. Das bedeutet, es wurden Komponenten entfernt, die nicht mehr benötigt wurden. Der Rest wurde in einer Datei zusammengefasst. Das erleichtert zum einen das Einbinden von diesen in anderen Dateien, zum anderen findet man sie leichter. Das Gleiche wurde mit kleinen Systemen gemacht, wie beispielsweise dem AIsystem, dem Pflanzensystem und den beiden Gießsystemen.  

\section{Multiagenten System}

\subsection{Sate Machines}

Wir haben noch keine State Machines für den Roboter implementiert, weil der Roboter sich nur in den zwei Zuständen \"Pflanzen gießen\" und \"Mit Spieler reden\" befinden kann. Diese Funktionalität kann über ein einfaches if-else abgedeckt werden und deswegen sind State Machines noch nicht nötig.

\subsection{Behaviour Tree}
Um den Behaviour Tree für den Handlungsablauf des Roboters effizient umzusetzen, wurde die Bibliothek BehaviorTree.CPP genutzt. Diese wurde ausgesucht, da sie vcpkg unterstützt, für C++ entwickelt wurde und alle nötigen Funktionalitäten bietet, ohne zu komplex zu sein. 
Im Falle des Gießroboters gibt es fünf Schritte, die er ausführen muss, um eine Pflanze zu gießen. Zuerst muss eine Pflanze ausgewählt werden, die am wenigsten Wasser hat. Im Anschluss wird der Weg von der aktuellen Position zu dieser gesucht. Daraufhin fährt der Roboter den Weg ab. Dort angekommen muss er sich noch so drehen, dass sich sein Gießarm über dem Blumentopf befindet, damit am Ende die Pflanze gegossen werden kann. Alle diese Schritte wurden in dem Behaviour Tree als Blattknoten implementiert. Darüber sitzt ein sogenannter Sequenzknoten, dieser merkt sich welches Blatt als letztes ausgeführt wurde und ob dieses beendet wurde. Wenn es noch nicht fertig ist, wird es im nächsten Schritt erneut aufgerufen. Dies geschieht so lang, bis das Blatt fertig ist. Der Sequenzknoten erhält als Rückmeldung „SUCCES“, oder, wenn das Blatt noch nicht fertig ist, „RUNNING“. Wenn ein Blatt fertig ist, wird das nächste ausgeführt. Wurde jedes Blatt einmal ausgeführt und hat jeweils „SUCCES“ zurückgegeben, so beginnt der Sequenzknoten erneut das erste Blatt aufzurufen. 
Die Baumstruktur wird in einem XML-Dokument gespeichert, die Funktionalität der einzelnen Knoten befindet sich in einer C++ Datei und die Knoten und Blätter werden in einer sogenannten Treefactory erstellt, welche ebenfalls in C++ geschrieben ist.
\subsection{Steering Behaviour}
Auch die Bewegungssteuerung wurde auf Basis des ECS implementiert. Dazu wurden ein Fahrsystem und Komponente, wie ein Fahrkomponente, Hinderniskomponente, Pfadkomponente und eine Positionskomponente, erstellt. Das Fahrsystem untersucht dabei, welcher Bewegungsmodus aktuell ausgewählt ist und passt darauf basierend das Bewegungsverhalten an. Es wird zwischen dem Pfadverfolgen und Drehen unterschieden. Ist Pfadverfolgen aktiviert, so folgt der Roboter einem Pfad, welcher sich in der Pfadkomponente befindet, die dem Entity Roboter zugeordnet wird. Der Pfad ist dabei eine Liste von Wegpunkten, die der Agent abfahren soll. Dazu wird das, wie im Konzeptteil beschriebene, Suchverhalten genutzt. Zudem wird, während der Roboter die Punkte abfährt, darauf geachtet, dass er mit keinen Hindernissen kollidiert. Dazu zählen alle Objekte in der Szene, welche eine Hinderniskomponente haben. In dieser ist der Radius des Gegenstandes oder Spielers gespeichert. Damit das Fahrverhalten des Roboters sichtbar ist, muss die Position in der Positionskomponente nach jedem Durchlauf aktualisiert werden. Das Ausweichverhalten wurde, wie im Konzeptteil beschrieben, implementiert. Das bedeutet, es wird überprüft, ob das Objekt, welches am nächsten zu dem Roboter ist, sich vor dem Agenten befindet. Ist dies der Fall, so wird geschaut, ob der Roboter nach dem nächsten Schleifendurchlauf mit diesem kollidiert, wenn ja wird der aktuelle Wegpunkt nach rechts oder links verschoben. Es wird dabei berücksichtigt, bei welcher Verschiebung die Abweichung zum Pfad am geringsten ist. Je nach dem wird der Pfad angepasst. Da der Punkt nun eine neue Position hat, weicht der Roboter dem Hindernis aus.
Soll der Roboter sich nun drehen, um beispielsweise mit dem Spieler zu reden, oder um eine Pflanze zu gießen, muss in der Fahrkomponente der aktuelle Modus auf Drehen gesetzt werden. In ihr sind zudem Informationen über die Masse, die Höchstgeschwindigkeit, die maximale Lenkkraft und die Größe des Sicherheitsabstandes gespeichert. Bei der Implementierung des Drehverhaltens, werden jedoch keine weiteren Punkte bestimmt, die der Agent anschauen soll, sondern der Drehwinkel wird interpoliert. Das bedeutet, dass der Wert für diesen reduziert wird, sodass der Roboter sich unabhängig von der Framerate langsam in die richtige Richtung dreht und zum Schluss korrekt ausgerichtet ist. Alle Objekte, die sich nun in dem Projekt bewegen sollen, erhalten eine Fahrkomponente, aktuell sind dies jedoch nur die zwei Gießroboter. 
Die Bewegungssteuerung hat jedoch noch zahlreiche Probleme. Da die Steuerung des Roboters auf einfachen mathematischen Formeln basiert, verfügt der Roboter nur über ein stark eingeschränktes Wissen über die Umgebung. Aus diesem Grund kann er zwar einem Hindernis ausweichen, es wird jedoch dabei immer nur ein Objekt berücksichtigt. Das bedeutet, dass er beispielsweise rechts an einem Hindernis vorbeifährt und damit auf einen Kollisionskurs mit einem anderen Gegenstand gerät, was nicht passiert wäre, wenn er dem ersten Hindernis auf der linken Seite ausgewichen wäre. Hätte der Roboter ein besseres Verständnis über die aktuelle Situation, so könnte er sich optimaler in der Welt bewegen. Zudem wird bei dem Ausweichverhalten nicht zwischen Objekten unterschieden, die stillstehen, oder sich bewegen. Es könnte passieren, dass er einen Bogen nach rechts fährt, um einem Objekt auszuweichen, dieses sich jedoch so bewegt, dass es trotz Ausweichmanöver immer noch im Weg befindet. Im Schlimmsten Fall könnte dieses Hindernis sich aufgrund seiner Bewegung dauerhaft zwischen Roboter und Ziel befinden. In dieser Situation wäre es für ihn mit dem aktuell implementierten Bewegungssystem unmöglich die Pflanze zu erreichen. Würde der Roboter ein besseres Verständnis über die Umgebung besitzen, so könnte er einfach nach links ausweichen.
Des Weiteren kann auf Grund der Simplizität nicht garantiert werden, dass der Roboter, während er auf eine Pflanze zufährt, nicht auf eine Kreisbahn um diese gelangt. Dies würde geschehen, wenn das Ziel sich links beziehungsweise rechts von dem Roboter befindet und er eine Kurve fahren muss, um es zu erreichen. Wenn die Krümmung der Kurve stärker ist als der Roboter mit seinem maximalen Lenkwinkel fahren kann, ist es für ihn unmöglich diesen Pfad zu fahren. Im schlimmsten Fall befindet sich das Ziel im Mittelpunkt der Kurve, die der Roboter aktuell fährt. Dies hätte zur Folge, dass sich der Abstand zur Pflanze nicht verändert und er sie nicht erreichen kann. Das aktuell implementierte Bewegungssystem kann nicht erkennen, ob er auf einer Kreisbahn ist. Daher können auch keine Korrekturzüge ausgeführt werden.
Außerdem ist die Bewegungssteuerung nicht intuitiv, da Eigenschaften, wie die Masse des Roboters oder seine maximale Krafteinwirkung dieses beeinflussen. Es ist dementsprechend bei der Erstellung neuer Roboter schwierig passende Werte für diese Merkmale zu finden, um das gewünschte Fahrverhalten zu erhalten. 
Während der Programmierung sind zahlreiche Herausforderungen aufgetreten. Durch sorgfältiges Analysieren und Zusammenarbeiten konnten jedoch alle Probleme im Verlauf des Projekts bewältigt werden.
\section{Partikelsystem}
Um das Partikelsystem in dem Projekt umzusetzen, wurden zwei neue Komponenten mit zusätzlich zwei neuen Systemen implementiert. Die Komponenten waren dabei eine Emiter- und eine Partikelkomponente. Die Emiterkomponente enthält Informationen über die Anzahl der Partikel, die pro Iteration erschaffen werden und die relative Position, wo dies geschehen soll. In dem Projekt muss berücksichtigt werden in welche Richtung der Roboter schaut, damit die Wassertropfen auch wirklich an der Stelle erscheinen, wo sich der Arm des Roboters befindet. In der Partikelkomponente wird die sogenannte Lebensdauer gespeichert, damit die Tropfen nach einer definierten Zeit auch wieder verschwinden bzw. entfernt werden können. Das ist sehr wichtig, da sonst immer mehr Entities erschaffen werden, was die Performance stark reduzieren würde und im schlimmsten Fall das Programm zum Absturz bringen kann. Das Partikelsystem erschafft Objekte mit einer Partikelkomponente, Geometriekomponente und einer Positionskomponente. Die Geometriekomponente ist dabei wichtig, um den einzelnen Tropfen ein Modell zuzuordnen, anderenfalls wäre der Benutzer nicht in der Lage das Wasser zu sehen. Bei der Erstellung der Entities wird zudem darauf geachtet, dass die Position, an der sie erschaffen werden, leicht variiert. Dadurch entsteht ein breiterer Strahl aus Tropfen, was das Gießen realistischer darstellt. Das zweite System ist das PartikelRemovalSystem. Dieses reduziert bei jeder Iteration den Wert der Lebensdauer aller Tropfen, bis dieser kleiner gleich Null ist. In diesem Fall wird dann das entsprechende Objekt entfernt. 
\section{Navigationssystem}

Um einen Pfad für ein Entity zu finden, muss man dem Entity eine PathRequestComponent mit Start und Ziel hinzufügen. Das Navigationssystem ließt die Daten aus dem Request aus, entfernt diesen und fügt stattdessen eine PathComponent mit dem gefundenen Pfad hinzu.

Um einen Pfad von einem Startpunkt zu einem Zielpunkt zu finden, sind drei Schritte nötig. Zuerst muss man die Navigationsnetzpolygone finden, auf denen Start und Ziel liegen. Danach sucht man die Polygone, die Start- und Zielpolygon verbinden und zum Schluss kann man diese Liste an Polygonen zu einem Pfad aus Punkten umwandeln.

\section{Physiksystem}

Die ECS-Bibliothek Flecs unterstützt Entity Event Listener, sogenannte Observer. Man kann eine Callbackfunktion definieren die gerufen wird, wenn eine Komponente eines Entities hinzugefügt, modifiziert oder entfernt wird. Das Physiksystemsystem nutzt zwei Obsover: ein Observer überwacht, wenn eine Physikkomponente zu einem Entity hinzugefügt wird und fügt den enthaltenen Rigidbody auch in die Physikwelt ein. Der zweite Observer wird gerufen, wenn ein Entity aus der Welt entfernt wird, welches eine Physikkomponente besitzt. In dem Fall wird auch der dazugehörige Rigidbody aus der Physikwelt entfernt.

Theoretisch benötigen wir nur die Kollisionserkennung und -auflösung von Bullet. Aber es war deutlich schwerer diese beiden Funktionen getrennt von der Physiksimulation zu rufen als die Physikwelt und die ECS-Welt zu synchronisieren. Deswegen funktioniert die Kollisionserkennung und -auflösung jetzt in drei Schritten. Zuerst werden alle Entities mit Physikkomponente und Transformationskomponente gesucht und die Geschwindigkeit und Position der Transformationskomponente auf den Rigidbody der Physikkomponente übertragen. Dann wird die Simulation der Physikwelt mit dem jetzigen Zeitschritt gerufen. Zum Schluss wird die Geschwindigkeit und Position des Rigidbodies wieder in die Transformationskomponente geschrieben.

Die Platformen, Zäune und Wendeltreppe sind statische Geometrie und zusätzlich noch Konkav. Deswegen haben sind diese mit einem Dreieckskollider repräsentiert. Dabei wird einfach das 3D-Modell der statischen Geometrie als Collider benutzt.

Alle Entities haben Kapsel Collider anstatt Zylinder Collider, da diese an den Dreieckskanten der statischen Geometrie hängen geblieben sind.

\section{Szenen Editor}

Die Assimp Variante hat nicht funktioniert, weil die Eingriffe zu tief in der Engine vorgenommen werden mussten. Außerdem war das Abstraktionslevel zu gering: wenn man nur Knoten hat ist es schwer Entities zu erkennen.

Das Blender Plugin wurde in Python programmiert und man hat Zugriff auf alle Optionen, auf die man auch im Editor Zugriff hat. Um jetzt die Entities zu exportieren, wurde über alle Objekte der Szene iteriert und überprüft, ob diese gelinkt sind. Wenn das der Fall ist, dann wurde die Transformation, der Name und der Name der Modelldatei in eine JSON-Datei geschrieben. Um jetzt auch noch die statische Geometrie zu exportieren, wurden alle gelinkten Objekte unsichtbar gemacht und nur die sichtbaren Modelle wurden exportiert. Alle unsichtbaren Objekte wurden anschließend wieder sichtbar gemacht.

In CrossForge wurde dann die JSON-Datei geladen und für jeden Eintrag wurde ein entsprechendes Entity zu Welt hinzugefügt.

\section{Dialogsystem}

Die gesamte Implementierung des Dialogsystems haben wir (Lisa und Sophie) im Pair-Coding übernommen. 

\subsection{Imgui}

Die Bibliothek für Imgui konnte über vcpkg recht schnell und einfach in das Projekt eingebunden werden. Zur Erzeugung von Dialogfenstern erstellen wir mit Imgui Frames, in denen oben der aktuelle Text des Dialogs angezeigt wird und unten die möglichen Antworten als Buttons. Wir nutzen außerdem einige Style-Anpassungen, um den Dialog etwas anschaulicher für den Nutzer zu gestalten, sodass z.B. das Fenster immer im Viewport zentriert ist.

\subsection{JsonCpp}

Ähnlich wie Imgui konnten wir auch JsonCpp über vcpkg einbinden. Mithilfe dieser Bibliothek können wir dann den Dialoggraphen initialisieren. Dafür sollten die Json-Dateien die gleiche Baumstruktur aufweisen, wie der Dialoggraph selbst. Über JsonCpp werden die Informationen aus der Datei gelesen und gespeichert, sodass sie dann weiter genutzt werden können, um den Dialoggraph zu füllen.

\subsection{Dialoggraph}

Den Dialoggraph haben wir so wie im Konzeptteil bechrieben implementiert. In einer dafür erstellten Klasse werden die zuvor genannten Daten gespeichert: der Dialogtext, ein Boolean für die Unterscheidung zwischen Roboter und Spieler und die möglichen Antworten. Zusätzlich gibt es noch eine Funktion, die die Initialisierung des Baumes umsetzt und eine Funktion für das Ersetzen der Strings mithlfe der Dialogmap.

\subsection{Dialogmap}

Mit der implementierten Dialogmap kann eine Funktion aufgerufen werden, die den Name einer Pflanze zurückgibt. Der Rückgabewert dieser Funktion ist allerdings statisch festgelegt, da zum Zeitpunkt der Abgabe das Dialogsystem noch nicht in das ECS eingebunden ist. Diese Integration wäre aber notwendig um die Positionen von Roboter und Spieler zu ermitteln, die gebraucht werden um zu bestimmen welcher Pflanzenname zurückgegeben werden soll. Auch die Pflanzennamen selbst sind im ECS gespeichert, weshalb ein Zugriff darauf vom Dialogsystem aus aktuell nicht möglich ist.  
Weitere Funktionalitäten der Dialogmap, wie Eventhandling sind ebenfalls noch nicht implementiert.

\section{Modelle und Szene}

Wie bereits im >Konzept< beschrieben ging die Anfangszeit in die Konzeptbearbeitung hinein. Ebenso hat es etwas Zeit in Anspruch genommen sich in Blender einzuarbeiten. Alle Modelle wurden mehrmals modelliert aufgrund von verschiedenen Problemen, wie z.B. Designunklarheiten, zu viele Polygone innerhalb des Models aufgrund zu vieler Details oder falscher Vorgehensweise, oder anderen Extras etc. \includegraphics[height=0.3\pageheight,keepaspectratio]{pics/1} ersten robomodels, ersten blätter mit wireframe<
Die Modelle wurden via Blender hergestellt und die jeweils dazugehörigen Texturen wurden selbstständig auf dem iPad mit ProCreate gezeichnet.
Der Roboter wurde grundsätzlich aus einfachen Zylindern modelliert und entsprechend angepasst. Die Grundüberlegung bei diesem war es, dass er nicht zu klassisch ‚standartrobotermäßig‘ aussieht, da man mit ihm kommunizieren kann. Die Solarplatte auf seinem Kopf wurde mit der Intension bzw. Überlegung integriert, dass er sich über die Sonne aufladen kann. Der Arm soll einen Schlauch darstellen, mit dem die Pflanzen gegossen werden können. Alle einzelnen Objekte, bis auf den Schlauch, wurden zum Schluss zusammen gemerged. Der Schlauch blieb als einzelnes Objekt, da er im späteren Verlauf animierbar sein soll. Was die Texturen angeht, wurden bis auf die Solarplatte, alle direkt in Blender via Geometry Nodes eingestellt. Daher bekam der Körper den metallischen, und die Räder den gummiartigen Look. Die Solarplatte wurde in ProCreate selbst gezeichnet und im UV-Editor eingefügt und angepasst. Nach Absprachen mit dem Team entstand dann unser Roboter wie man ihn aktuell im MVP sehen kann. >BILD EINFÜGEN<
Die beiden Pflanzen sind relativ frei aus dem Kopf entstanden. Die einzigen Überlegungen dabei waren es, zwei unterschiedliche Pflanzen zu haben. Bei beiden entstanden die Blumentöpfe aus einfachen Zylindern. Der Stiel der kleineren Pflanze entstand anfangs erst aus einem einfachen Mesh mit verschiedenen Einstellungen der Geometry Nodes [1]. Diese Variante wurde allerdings verworfen und dann doch vereinfacht. Die Blätter, welche anfangs relativ 3D-getreu, aber zu detailliert waren bestehen jetzt nur noch aus zwei einfachen 2D-Texturen. Alle Texturen, sowohl Blätter als auch die Töpfe mit Erde, wurden in ProCreate selbstständig erstellt. Der erste versuch bestand daraus, die Modelle in ProCreate zu importieren und direkt darauf zu zeichnen. Diese Texturen hätten dann in Blender ganz einfach importiert werden können. Allerdings funktionierte diese Variante nicht ganz so wie erhofft, weswegen die Texturen dann einfach 2D gezeichnet wurden, und dann via UV-Mapping in Blender angepasst wurden. Wie auch bei dem Roboter wurden die Pflanzen nach Absprache mit dem Team abgesegnet und ins MVP aufgenommen. >BILD EINFÜGEN<
Die größere Hürde waren die Plattformen und deren Design. Wie bereits im Konzept beschrieben, entstanden da anfangs verschiedene Ideen. Beide Plattformen entstanden letzten Endes frei Hand und dazu eine etwas ausgefallenere Verbindung zwischen eben diesen Beiden. Damit die Roboter sich im Notfall auch zwischen den beiden Plattformen bewegen können, entstand dazu mit Inspiration noch eine Rampe um die Säule herum [2]. Damit sowohl die Rampe als auch die Plattformen etwas sicherer sind, bekamen diese noch jeweils einen Zaun und ein Geländer herum. Auch hier entstanden alle Texturen via ProCreate aus eigener Hand. Dieses Modell wurde ebenso mit dem Team abgesprochen, bevor es final war. >BILD EINFÜGEN< Am Ende wurde leider etwas zu spät aufgefallen, dass die Rampe breiter sein sollte, für den Fall, dass beide Roboter gleichzeitig die Ebenen wechseln wollen und sich auf der Rampe treffen. Zum aktuellen Zeitpunkt ist dort leider kein Platz zum Ausweichen und einer der Roboter driftet ab.
Nachdem alle Modelle fertig waren, wurden diese in der Szene zusammengesetzt und weitestgehend angepasst. Das Ganze wurde dann mithilfe eines Dropper Addon ins Projekt geladen. Hier gab es Anfangs einige Probleme und Unklarheiten, welche aber letzten Endes mit Hilfe des Teamleiters gelöst wurden. Um die Zusammenarbeit zwischen den Modellen und den restlichen Programmieraufgaben, wie das Navigationssystem, zu verbessern, wurden die Modelle in ‚linked‘ (Pflanzen und Roboter) und ‚static‘ (Plattformen & co) eingeteilt.
Als die Szene an sich so weit fertig war und auch im Projekt geladen war, gab es nur noch ein paar generelle Probleme zu lösen, was Texturen und einige Ordnerstrukturen anging.
Die nächste Aufgabe galt dann eigentlich die MVP-Szene weiter auszubauen und aufzuhübschen. Hierfür entstanden über ein neues Addon zwei Bäume und ein Haus. Auch hier wurden sowohl für die Bäume als auch für das Haus alle Texturen über ProCreate erstellt, wobei das Haus noch nicht ganz fertig war. >BILD EINFÜGEN< Aufgrund des Zeitlimits wurde der Ausbau der Szene allerdings vorzeitig abgebrochen.


% -- -- -- -- -- -- -- -- -- -- -- -- -- -- -- -- -- -- -- -- -- -- -- -- -- -- -- -- -- -- -- -- --
\subsection{Ergebnisse}
% vim: set ts=4 sw=4 smartindent expandtab textwidth=100:

Jede Arbeit hat normalerweise Ergebnisse. Dies können Messreihen, Beweise und vieles mehr sein. In diesem Kapitel werden die Ergebnisse präsentiert und diskutiert. Meist ist die Implementation nicht vollkommen und zeigt in Randbereichen Schwächen. Hier ist der Platz dies aufzuzeigen.


% -- -- -- -- -- -- -- -- -- -- -- -- -- -- -- -- -- -- -- -- -- -- -- -- -- -- -- -- -- -- -- -- --
\subsection{Zusammenfassung \& Ausblick}
% vim: set ts=4 sw=4 smartindent expandtab textwidth=100:

Die Zusammenfassung ist häufig das erste, was nach dem Titel einer Arbeit gelesen wird. Es sollte ein kurzer (circa eine Seite) Überblick über das Erreichte gegeben werden ohne sich in Details zu verlieren.

Im Ergebniskapitel wurden vielleicht Schwächen der Implementierung oder auch des Konzeptes aufgezeigt. Im Ausblick können hier nun Lösungsmöglichkeiten aufgezeigt werden, die sich im Verlauf der Bearbeitung nicht umsetzen ließen. Es sollten fundierte Lösungskonzepte erarbeitet werden. Weiterhin können anschließende Arbeiten in angrenzenden Gebieten vorgeschlagen werden.


% -- -- -- -- -- -- -- -- -- -- -- -- -- -- -- -- -- -- -- -- -- -- -- -- -- -- -- -- -- -- -- -- --
\subsection{Nachsätze}
% vim: set ts=4 sw=4 smartindent expandtab textwidth=100:

Im Anhang kommen die Sachen unter, die in der Arbeit keinen Platz haben. Hier finden sich eventuell ausführliche Algorithmenbeschreibungen. Aber auch Ergebnisse, die in der Arbeit sonst überflüssig wären, weil sie beispielsweise das gleiche zeigen wie die restlichen, können hier untergebracht werden.

\section{Arbeitsaufteilung}

\begin{itemize}
\item Angelique: Modelle
\item Lisa und Sophie: Dialogsystem
\item Leon:
\begin{itemize}
\item Anbindung von Flecs
\item Implementierung von Multiagentensystem
\item Konzept und Implementierung Partikelsystem
\end{itemize}
\item Linus:
\begin{itemize}
\item Projekt Management
\item Architektur
\item Konzept Multiagentensystem
\item Konzept und Implementierung von Navigationssystem, Physiksystem, Szenen Editor
\end{itemize}
\end{itemize}

\printbibliography

\end{document}
