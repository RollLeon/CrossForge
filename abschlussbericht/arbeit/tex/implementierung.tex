% vim: set ts=4 sw=4 smartindent expandtab textwidth=100:

\section{Entity Component System}

\section{Multiagenten System}

\subsection{Sate Machines}

nicht mehr dazu gekommen, weil Zeit vorbei war und Aufgaben von anderen Projektteilnehmern übernommen werden musste
aber auch nicht nötig, weil Dialogsystem noch nicht fertig war

\subsection{Behaviour Tree}

\subsection{Steering Behaviour}

\section{Partikelsystem}

\section{Navigationssystem}

Keine Besonderheiten. Ist halt so wie im Konzept

\section{Physiksystem}

Da das Physiksystem mit Observern feststellt, ob eine Physikkomponente hinzugefügt wurde, muss die Komponente schon initialisiert sein und man kann die lazy Initialisierung nicht benutzen. Deswegen mit emplace statt add.

Die Platform und Wendeltreppe ist statische Geometrie und zusätzlich noch Konkav. Deswegen haben wir diese mit einem Dreieckskollider ausgestattet. Dabei wird einfach das 3D-Modell als Kollider benutzt.

Alle Entities haben Kapsel Collider, da die Entities sonst an den Dreieckskanten der statischen Geometrie hängen geblieben sind.

\section{Szenen Editor}

Die Assimp Variante hat nicht funktioniert, weil die eingriffe zu tief in der Engine vorgenommen werden mussten. Außerdem war das Abstraktionslevel zu gering: wenn man nur Knoten hat ist es schwer Entities zu erkennen.

Das Blender Plugin wurde in Python programmiert und man hat zugriff auf alle Optionen, auf die man auch im Editor Zugriff hat. Um jetzt die Entities zu exportieren, wurde über alle Objekte der Szene iteriert und überprüft, ob diese gelinkt sind. Wenn das der Fall ist, dann wurde die Transformation, der Name und der Name der Modelldatei in eine JSON-Datei geschrieben. Um jetzt auch noch die statische Geometrie zu exportieren, wurden alle Entities unsichtbar gemacht und nur die sichtbaren Modelle wurden exportiert.

In CrossForge wurde dann die JSON-Datei geladen und für jeden Eintrag wurde ein entsprechendes Entity zu Welt hinzugefügt.

\section{Dialogsystem}

\section{Management}

- Zusammensetzung
.. generell nach versucht aufgaben nach stärken der Personen zu verteilen
.. Angelique bricht Informatik Studim ab -> bekommt kreative Modellierungsaufgaben
.. Lisa und Sophie haben bis jetzt weniger Erfahrung und bekommen leichtere Aufgaben
.. Leon hat gewisse leidensfähigkeit und bekommt mittlere schwere Aufgaben
.. Carlo und Linus haben meisten Erfahrungen, arbeiten neben dem Studium und bekommen schwersten Aufgaben

- Arbeisweise:
.. wir haben uns auf 6h/Woche geeinigt, da Arbeit/Studium war nicht mehr drin
.. Jira am Anfang angelegt, nicht wieder benutzt
.. gegen Scrum entschieden, weil Retros/Plannings/Sprintwechsel zu viel overhead sind und wir leichtgewichtigen prozess brauchen
.. Linus hat mit Lisa, Sophie, Leon eine Stunde in der Woche Pair Codings gemacht, um bei den Aufgaben zu helfen
.. halbzeit haben wir eine retro gemacht
.. retro hat festgestellt, dass wir verschiedene Arbeitsweisen mögen: Lisa und Sophie bringen Pair Codings weniger, Leon schätzt sie sehr und würde gerne fortfahren

- Angelique
.. eigenständige einarbeitung in Blender
.. völlig autark und angenehmes management: ich hab aufgaben rüber geworfen und die wurden erledigt
.. hat mir Zeit erspart

- Dialogsystem wurde nicht fertig
.. generell darauf geachtet, dass stände kompatibel bleiben (09.10.2023 merge aller branches in den master und dann den master branch in alle feature branches)
.. somit wäre es möglich ab sync merge Dialogsystem in ECS zu integrieren
.. aber Lisa und Sophie waren Oktober und November mit Aufräumarbeiten und Anzeigen, dass man eine Interaktion starten kann beschäftigt. Somit hatten sie keine Zeit, das Dialogsystem in ECS zu integrieren
.. Problem liegt hier bei unterschiedlicher Einschätzung von schwierigkeit und dauer. Für eine änderung, für die ich ein paar Stunden brauchen würde, haben die beiden mehrere Wochen benötigt. Den Verlauf kann man auch an der Git Historie sehen.


- Carlo hat das Praktikum abgebrochen
.. meiste CPP Erfahrung und arbeitet 18h pro Woche -> meiste Erfahrung und somit Aufgabe mit größten unsicherheiten bekommen
.. Carlo hatte Aufgabe Detour in CrossForge zu integrieren
.. Carlo hat aufgrund von Arbeit, Uni, Umzug, Problemen mit seiner Entwicklungsumgebung nur langsamen Fortschritt mit seiner Aufgabe was verständlicherweise zu frustration führt
.. bei Retro hat er Angebot bekommen eine andere Aufgabe zu übernehmen -> hat er nach mehreren Wochen ohne nennenswerte Fortschritte angenommen
.. nächste Aufgabe war Partikelsystem, Linus übernimmt Detour integration -> wieder IDE Probleme und kein Fortschritt
.. Lust am programmieren verloren -> nächste Aufgabe war mit Angelique zusammen Modelle (Aufgabe vor Weihnachten erteilt), Leon übernimmt Partikelsystem
.. über Jahreswechsel hat niemand von uns am Projekt gearbeitet und da Leon am Anfang Februar nach Polen fährt, muss das Projekt bis dahin abgeschlossen sein und die volle Konzentration sollte jetzt auf Bericht und Präsentation liegen -> keine neuen Modelle/Szenen
.. somit hat Carlo nur extrem wenig beigetragen und somit entschieden, dass er das TP abbricht

.. mehr unterstützung und klein geschnittene Aufgaben wären hilfreich gewesen, konnte ich aber nicht machen, weil ich mich dafür so sehr in die Materie einlesen müsste, dass ich das Problem auch selber lösen kann. Ich hatte dafür auch keine Zeit, weil ich mit den restlichen Aufgaben und Teilnehmern überfordert war.

